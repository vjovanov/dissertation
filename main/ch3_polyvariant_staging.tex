\chapter{Polyvariant Staging}


\emph{Multi-stage programming} (or \emph{staging}) is a meta-programming technique
  where compilation is separated in multiple \emph{stages}. Execution of each
  stage outputs code that is executed in the next stage of compilation. The first
  stage of compilation happens at the \emph{host language} compile time, the second
  stage happens at the host language runtime, the third stage happens at runtime of
  runtime generated code, etc. Different stages of compilation can be executed in the same
  language~\cite{taha_multi-stage_1997,nielson2005two} or in different languages~\cite{brown_heterogeneous_2011,devito2013terra}.
  In this work we will focus on staging systems where all stages are in the same language and that, through static typing, assure
  that terms in the next stage are well typed.

  Notable staging systems in statically typed languages are
  MetaOCaml~\cite{taha_multi-stage_1997,calcagno2003implementing}
  and LMS~\cite{rompf2012lightweight}. These systems were successfully applied as a
  \emph{partial evaluatior}~\cite{jones1993partial}: for removing abstraction
  overheads in high-level programs~\cite{carette2005multi,rompf2012lightweight},
  for domain-specific languages~\cite{czarnecki_dsl_2004,jonnalagedda2014staged,taha2004gentle}, and for converting language
  interpreters into compilers~\cite{lancet,futamura1999partial}. Staging originates
  from research on two-level~\cite{nielson2005two,davies1996temporal} and multi-level~\cite{davies1996modal} calculi.

 We show an example of how staging is used for partial evaluation of a function
 for computing the inner product of two vectors\footnotemark[1]:\begin{lstparagraph}
def dot[T:Numeric](v1: Vector[T], v2: Vector[T]): T =
  (v1 zip v2).foldLeft(zero[T]) {
    case (prod, (cl, cr)) => prod + cl * cr
  }
 \end{lstparagraph}

In function \code{dot}, if vector sizes are constant, the inner product can
 be partially evaluated into a sum of products of vector components. To achieve partial evaluation,
 we must communicate to the staging system that operations on values of vector components
 should be executed in the next stage. The compilation stage
 in which a term is executed is determined by \emph{code quotation}~(in MetaOCaml)
 or by parametric types \code{Rep}~(in LMS). In LMS marking
 that the vector size is statically known is achieved by annotating only vector elements with
 a \code{Rep} type\footnotemark[2]:\begin{lstparagraph}
def dot[T:Numeric]
  (v1: Vector[Rep[T]], v2: Vector[Rep[T]]): Rep[T]
 \end{lstparagraph}

Here the \code{Rep} annotations on \code{Rep[T]} denote that elements of vectors will be known
 only in the next stage (in LMS, this is a stage after run-time compilation). After run-time compilation \code{zip},
 \code{foldLeft}, and pattern matching inside the closure will not exist
 as they were evaluated in the previous stage of compilation (host language runtime). Note that in
 LMS unannotated code is always executed during host-language runtime
 and type-annotated code is executed after run-time compilation.

{\bf Staging at host language compile time.} How can we use staging for programs whose values are statically
 known at the host language compile-time (the first stage)? Existing staging frameworks
 treat unannotated terms as runtime values of the host language and annotated terms as
 values in later stages of compilation. Even if we would take that the first stage is executed
 at the host language compile time, we would have to annotate all run-time values.
 Annotating all values is cumbersome since host language run-time values comprise
 the majority of user programs~(\sct{sct:discussion}).

\footnotetext[1]{All code examples are written in \emph{Scala}. It is necessary to
know the basics of Scala to comprehend this paper.}
\footnotetext[2]{In this work we use LMS as a representative of type-based staging systems.}

MacroML~\cite{ganz2001macros} expresses macros as two-stage computations that start executing from host language compile time.
 In MacroML, parameters of macros can be annotated as an early stage computation. These parameters
 can then used in escaped terms for compile-time computation. Terms scheduled for runtime execution,
 withing the escaped terms, again need to be quoted with brackets. This,
 in effect, imposes quotation for both escaping and brackets which requiring additional effort.

% Type annotations, and thus all operations continue to work!
{\bf Code Duplication.} Staging systems based on type annotations (e.g., LMS and type-directed
partial evaluation~\cite{danvy1999type}) inherently require code duplication as,
a priory, no operations are defined on \code{Rep} annotated types. For example,
in the LMS version of the \code{dot} function, all numeric types (\ie, \code{Rep[Int]}, \code{Rep[Double]}, etc.)
must be re-implemented in order to typecheck the programs and achieve code generation
for the next stage.

Sujeeth et al.~\cite{forge} and Jovanovic et al.~\cite{yin-yang}
 propose generating code for the next stage computations based on
 a language specification. These approaches solve the problem,
 but they require writing additional specification for the libraries,
 require a large machinery for code generation,
 and support only restricted parts of Scala.

{\bf Annotating the Previous Stage.} The main idea of this paper is that \emph{annotated types}
 should denote computations that happen during the \emph{previous stage} of compilation.
 The reason is that static terms appear less frequently than run-time terms in a large set
 of analyzed programs (\sct{sct:discussion}). Therefore, annotating static terms
 introduces less overhead for the programmer.

% TODO count the number of terms
We treat annotated types as \emph{compile-time views}
 of existing data types. Compile-time view of a type denotes that all operations on that type are
 executed at host language compile time. We promote types to their compile-time views
 with the \code{@ct} annotation (\eg, \code{Int@ct}). Similarly, statically known terms can be promoted
 their compile time duals with the \code{ct} function on the term level.
 By having two views of the same type we obviate the need for introducing reification and code generation logic for
 existing types.

% Restricted to compile-time, but just techincal.
With compile-time views, to require that vectors \code{v1} and \code{v2} are
 static and to partially evaluate the function, a programmer needs to make
 a simple modification of the \code{dot} signature:\begin{lstparagraph}
def dot[V: Numeric@ct]
  (v1: Vector[V]@ct, v2: Vector[V]@ct): V
\end{lstparagraph}

Since, vector elements are polymorphic the result
 of the function can be a dynamic value, or a compile-time view
 that can be further used for compile-time computations. The binding time of the return type
 of \code{dot} will match the binding time of vector elements:

\begin{lstparagraph}
  // [el1, el2, el3, el4] are dynamic decimals
  dot(Vector(el1, el2), Vector(el3, el4))
    $\hookrightarrow$ (el1 * el3 + el2 * el4): Double

  // static terms are internally tracked
  dot(Vector(2.0, 4.0), Vector(1.0, 10.0))
    $\hookrightarrow$ (2.0 * 1.0 + 4.0 * 10.0): Double@static

  // ct promotes static terms to compile-time views
  dot(Vector(ct(2), ct(4)),
      Vector(ct(1), ct(10)))
    $\hookrightarrow$ 42: Double@ct
\end{lstparagraph}

In this paper we contribute to the state of the art:
\begin{itemize}

 \item By introducing compile-time views~(\sct{sct:interface}) as means to succinctly achieve
  type safe two-stage programming starting from host language compile time.

 \item By obviating the need for reification and code generation logic in type based staging systems.

 \item By demonstrating the usefulness of compile-time views in four case
  studies (\sct{sct:case-studies}): inlining, partially evaluating recursion,
  removing overheads of variable argument functions, and removing overheads of
  type-classes~\cite{wadler1989make,hall_type_1996,oliveira_type_2010}.

\end{itemize}

We have implemented a staging extension for Scala \tool \footnotemark[3].
 \tool has a minimal interface (\sct{sct:interface}) based on type annotations.
 We have evaluated performance gains and the validity of \tool on all case
 studies~(\sct{sct:case-studies}) and compared them to LMS. In all benchmarks (\sct{sct:evaluation})
 our evaluator performs the same as LMS and gives significant performance gains compared to original programs.

\footnotetext[3]{Source code: \url{https://github.com/scala-inline/scala-inline}.}

\section{Compile-Time Views in Scala}
\label{sct:interface}

In this section we informally present \tool, a staging extension for Scala based on compile-time views.
 \tool is a compiler plugin that executes in a phase after the
 Scala type checker. The plugin takes as input typechecked Scala programs and uses
 type annotations~\cite{odersky_1996_putting} to track and verify information about the biding-time
 of terms. It supports only two stages of compilation: host language compile-time
 (types annotated with \code{@ct}) and host language run-time (unannotated code).

To the user, \tool exposes a minimal interface (\figref{fig:interface}) with
a single annotation \code{ct} and a single function \code{ct}.

\begin{figure}
\begin{listingtiny}
package object scalact {
  final class ct extends StaticAnnotation

  @compileTimeOnly def ct[T](body: => T): T = ???
}
\end{listingtiny}
\label{fig:interface}
\caption{Interface of \tool.}
\end{figure}

\smartparagraph{Annotation \code{ct}} is used on types~(\eg,
\code{Int@ct}) to promote them to their compile-time views. The
annotation is integrated in the Scala's type system and, therefore, can be
arbitrarily nested in different variants of types.

Since all operations on compile-time views are executed at compile time, non-generic
 method parameters and result types of compile-time views also become compile-time views. Table \ref{tbl:ct-type}
 shows how the \code{@ct} annotation can be placed on types and how it affects method
 signatures on annotated types.

\begin{table*}[t]
\caption{Compile-time views of types and their corresponding method signatures.}
\label{tbl:ct-type}
\centering
\begin{tabularx}{\linewidth}{ X X X X }
\toprule

  Annotated Type              & \ &  Type's Method Signatures                          &  \\
  \code{Int@ct}               & \ &  \code{+(rhs: Int@ct): Int@ct}                     &  \\
  \code{Vector[Int]@ct}       & \ &  \code{map[U](f: (Int => U)@ct): Vector[U]@ct}     &  \\
                              & \ &  \code{length: Int@ct}                             &  \\
  \code{Vector[Int@ct]@ct}    & \ &  \code{map[U](f: (Int@ct => U)@ct): Vector[U]@ct}  &  \\
                              & \ &  \code{length: Int@ct}                             &  \\
  \code{Map[Int@ct, Int]@ct}  & \ &  \code{get(key: Int@ct): Option[Int]@ct}           &  \\

\bottomrule
\end{tabularx}
\end{table*}

 In \tabref{tbl:ct-type}, on \code{Int@ct} both parameters and result types of all
 methods are also compile-time views. On the other hand, \code{Vector[Int]@ct} has parameters
 of all methods transformed except the generic ones. In effect, this, makes higher order combinators of \code{Vector}
 operate on dynamic values, thus, function \code{f} passed to \code{map} accepts
 the dynamic value as input. Type \code{Vector[Int@ct]@ct} has all methods executed
 at compile-time. The return type of the function \code{map} on \code{Vector[Int@ct]@ct}
 can still be either dynamic or a compile-time view due to the type parameter \code{U}.

Annotation \code{ct} can be used to achieve simple inlining of statically
 known methods and functions. This is achieved by putting the annotation of the method/function
 definition:\begin{lstparagraph}
 def dot[V: Numeric]
  (v1: Vector[V], v2: Vector[V]): V
\end{lstparagraph}
Annotated methods will have an annotated method type\begin{lstparagraph}
((v1: Vector[V], v2: Vector[V]) => V)@ct
\end{lstparagraph} which can not be written by the users. This is not the first time
that inlining is achieved through partial evaluation~\cite{monnier2003inlining}.



\smartparagraph{Function \code{ct}} is used at the term level
 for promoting literals, modules, and methods/functions into their compile-time views.
 Without \code{ct} we would not be able to instantiate compile-time views of types.
 \tabref{tbl:ct-term} shows how different types of terms are promoted to their
 compile-time views. An exception for promoting terms to compile-time views is the
 \code{new} construct. Here we use the type annotation on the constructed type.


\begin{table*}[t]
\caption{Promotion of terms to their compile-time views.}
\label{tbl:ct-term}
\centering
\begin{tabularx}{\linewidth}{ X X }
\toprule

  Promoted Term        \quad \quad \quad & Term's Type                      \\
  \code{ct(Vector)(1, 2, 3)            } & : \code{ Vector[Int]@ct        } \\
  \code{ct(Vector)(ct(1), ct(2), ct(3))} & : \code{ Vector[Int@ct]@ct     } \\
  \code{ct((x: Int@ct) => x)           } & : \code{ (Int@ct => Int@ct)@ct } \\
  \code{ct((x: Int) => x)              } & : \code{ (Int => Int)@ct       } \\
  \code{new (::@ct)(1, Nil)            } & : \code{ (::[Int])@ct          } \\
  \code{new (::@ct)(ct(1), ct(Nil))    } & : \code{ (::[Int@ct])@ct       } \\

\bottomrule
\end{tabularx}
\end{table*}

\subsection{Tracking Binding-Time of Terms}
\label{sct:static}

 Internally \tool has additional type annotations for tracking the binding time of terms.
  Type of each term is annotated with either \code{dynamic}, \code{static}, or \code{ct}. \code{dynamic} denotes
  that the term can only be known at runtime, \code{static} that the term is known
  at compile-time but it will not be computed at compile time, and \code{ct} that
  the term will be computed at compile-time.

 Tracking static terms was studied in the context of binding-time analysis
  in partial evaluation for typed~\cite{nielson_1988_automatic} and
  untyped~\cite{gomard1991partial} languages. We use similar techniques, however,
  unlike in partial evaluation we do not evaluate static terms at compile time. They are tracked for verifying
  correctness and providing convenient implicit conversions. Static terms are evaluated only
  when they are explicitly marked by the programmer with \code{ct}.

  % What are the static terms
In \tool language literals, functions, direct class constructor calls with static arguments, and static method
 calls with static arguments are marked as static. Examples of static terms are\begin{lstparagraph}
1.0, "1", (x: Int => x), new Cons(1, Nil), List(1,2,3)
\end{lstparagraph}

\subsection{Least Upper Bounds}
\label{sct:lub}

 We use subtyping of Scala to simplify tracking of binding times by introducing a
 subtyping relation between \code{dynamic}, \code{static}, and \code{ct}. We argue that
 a \code{static} type is a more specific \code{dynamic} as it is statically known
 and that \code{ct} is more specific than \code{static} as its operations are executed
 at compile time. Therefore we establish that\begin{lstparagraph}
                 ct <: static <: dynamic
\end{lstparagraph}

 The use of subtyping simplifies tracking binding times of terms as in all cases
 where least upper bounds are calculated we can use the same mechanism for binding-times.
 An interesting example are the binding times of type parameters:\begin{lstparagraph}
ct(List)(1, ct(2)): List[Int@static]@ct
ct(List)(ct(1), ct(2)): List[Int@ct]@ct
ct(List)((x: Int@dynamic), ct(2)): List[Int@dynamic]@ct
\end{lstparagraph}

Notable exception are control flow constructs for which the original Scala rules for least
 upper bounds do not hold. The binding-time of control flow constructs does not
 depend only on the return type of the branches but also on the conditionals. For example, if
 both branches of an \code{if} construct are \code{static} the result can still be \code{dynamic}
 if the condition is \code{dynamic}. Here subtyping also helps as the binding type of the
 return value is simply calculated as \code{lub(c, thn, elz)} where \code{lub(tps: Type*)} is a function
 for computing the least upper bounds of types, and \code{c}, \code{thn}, \code{elz} are respectively
 binding times of the condition, the then branch, and the else branch. The same principles can be applied for
 pattern matching.

\subsection{Well-Formedness of Compile-Time Views}
\label{sct:wf-ctv}

% Nice description of csp and pointer to the right paper.
Earlier stages of computation can not depend on values from later stages. This property,
 defined as \emph{cross-stage persistence}~\cite{taha_multi-stage_1997,westbrook2010mint},
 imposes that all operations on compile-time views must known at compile time.

To satisfy cross-stage persistence \tool verifies that binding time of composite
 types~(\eg, polymorphic types, function types, record types, etc.) is always
 a subtype of the binding time of their components. In the following example,
 we show malformed types and examples of terms that are inconsistent:\begin{lstparagraph}
xs: List[Int@ct]     => ct(Predef).println(xs.head)
fn: (Int@ct=>Int@ct) => ct(Predef).println(fn(ct(1)))
\end{lstparagraph}

In the first example the program would, according to the semantics of \code{@ct}, print a head of the list at compile time.
 However, the head of the list is known only in the runtime stage. In the second example the program should
 print the result of \code{fn} at compile time but the body of the function will
 be known only at runtime. By causality such examples are not possible.

% Examples on classes
On functions/methods the \code{ct} annotation requires that function/method bodies are known at compile-time.
 Otherwise, inlining of such functions/methods would not be possible at compile-time. In Scala,
 method bodies are statically known in objects and classes with final methods, thus, the \code{ct}
 annotation is only applicable on such methods.

\subsection{Implicit Conversions}
\label{sct:implicits}

% Requires desugaring
If method parameters require compile-time views of a type the corresponding arguments
 in method application would always have to be promoted to \code{ct}.
 In some libraries this could require an inconveniently large number
 of annotations.

To minimize the number of required annotations we introduce implicit conversions from certain \code{static} terms to \code{ct} terms.
 The conversions support translation of language literals, direct class constructor calls with static arguments, and static method
 calls with static arguments into their compile-time views. Since our compile-time evaluator does
 not use Asai's~\cite{asai2002binding,sumii2001hybrid} method to keep track of
 the value of each static term, we disallow implicit conversions of terms with static variables.

For example, for a factorial function \begin{lstparagraph}
def fact(n: Int @ct): Int@ct =
  if (n == 0) 1 else fact(n - 1)
 \end{lstparagraph} we will not require promotions of literals \code{0}, and \code{1}. Furthermore,
 the function can be invoked without promoting the argument into it's compile-time view:\begin{lstparagraph}
fact(5)
  $\hookrightarrow$ 120
 \end{lstparagraph}

Without implicit conversions the factorial functions would be more verbose \begin{lstparagraph}
def fact(n: Int @ct): Int@ct =
  if (n == ct(0)) ct(1) else fact(n - ct(1))
 \end{lstparagraph} as well as each function application (\code{fact(ct(5))}).


\section{Case Studies}
\label{sct:case-studies}

In this section we present selected use-cases for compile-time views that, at the
same time, demonstrate step-by-step the mechanics behind \tool. We start by inlining a simple function with staging
(\sct{sct:inlining}), then do the canonical staging  example of the integer power function
(\sct{sct:recursion}), then we demonstrate how variable argument functions can
be  desugared into the core functionality (\sct{sct:varargs}). Finally, we
demonstrate how the abstraction overhead of the \code{dot} function and all
associated type-class related abstraction an be removed (\sct{sct:dot-product}).

\subsection{Inlining Expressed Through Staging}
\label{sct:inlining}

Function inlining can be expressed as staged computation~\cite{monnier2003inlining}.
 Inlining is achieved when a statically known function body is applied with symbolic
 arguments. In \tool we use the \code{ct} annotation on functions and methods to achieve inlining:\begin{lstparagraph}
@ct def zero[T](implicit num: Numeric[T]) = num.zero

zero[Double]
  $\hookrightarrow$ num.zero
\end{lstparagraph}


\subsection{Recursion}
\label{sct:recursion}

The canonical example in staging literature is partial evaluation of the power function
 where exponent is an integer:
\begin{lstparagraph}
def pow(base: Double, exp: Int): Double =
  if (exp == 0) 1 else base * pow(base, exp - 1)
\end{lstparagraph} When the exponent (\code{exp}) is statically known this function can be partially
evaluated into \code{exp} multiplications of the \code{base} argument, significantly
improving performance~\cite{calcagno2003implementing}.

With compile-time views making \code{pow} partially evaluated requires adding only one annotation:

\begin{lstparagraph}
def pow(base: Double, exp: Int@ct): Double =
  if (exp == 0) 1 else base * pow(base, exp - 1)
\end{lstparagraph}

% TODO cite infinite recursion
To satisfy cross-stage persistence (\sct{sct:wf-ctv}) the \code{pow} must be \code{@ct}.
However, to reduce the number of required annotations we implicitly add the \code{ct} annotation
when at least one parameter type or the result type is marked as \code{ct}. In the example
 the \code{ct} annotation on \code{exp} requires that the function must be called with
 a compile-time view of \code{Int}. \tool ensures that the definiton of the \code{pow} function
 does not cause infinite recursion at compile-time by invoking the power function
 only when the value of the \code{ct} arguments is known.

 The application of the function \code{pow} with a constant
 exponent will produce:

\begin{lstparagraph}
pow(base, 4)
  $\hookrightarrow$ base * base * base * base * 1
\end{lstparagraph}

Constant 4 is promoted to \code{ct} by the implicit conversions (\sct{sct:implicits}).

\subsection{Variable Argument Functions}
\label{sct:varargs}

% Variable argument functions
Variable argument functions appear in widely used languages like Java, C\#, and Scala.
 Such arguments are typically passed in the function body inside of the data structure
 (\eg \code{Seq[T]} in Scala). When applied with variable arguments the size of the
 data-structure is statically known and all operations on them can be partially
 evaluated. However, sometimes, the function is called with arguments of dynamic size.
 For example, function \code{min} that accepts multiple integers\begin{lstparagraph}
def min(vs: Int*): Int = vs.tail.foldLeft(vs.head) {
  (min, el) => if (el < min) el else min
}
\end{lstparagraph}can be called either with statically known arguments
 (\eg, \code{min(1,2)}) or with dynamic arguments:\begin{lstparagraph}
val values: Seq[Int] = ... // dynamic value
min(values: _*)
\end{lstparagraph}

Ideally, we would be able to achieve partial evaluation if the arguments are of statically
known size and avoid partial evaluation in case of dynamic arguments. To this end we translate
the method \code{min} into a partially evaluated version and a dynamic version. The call to these
methods is dispatched, at compile-time, by the \code{min} method which checks if
arguments are statically known. Desugaring of \code{min} is shown in \figref{fig:min}.

\begin{figure}
\begin{listingtiny}
def min(vs: Int*): Int = macro
  if (isVarargs(vs)) q"min_CT(vs)"
  else q"min_D(vs)"

def min_CT(vs: Seq[Int]@ct): Int =
  vs.tail.foldLeft(vs.head) { (min, el) =>
    if (el < min) el else min
  }

def min_D(vs: Seq[Int]): Int =
  vs.tail.foldLeft(vs.head) {
    (min, el) => if (el < min) el else min
  }
\end{listingtiny}
\caption{Function \code{min} is desugared into a \code{min} macro that based on the
binding time of the arguments dispatches to the partially evaluated version (\code{min_CT})
for statically known varargs or to the original min function for dynamic arguments \code{min_D}.}
\label{fig:min}
\end{figure}

\subsection{Removing Abstraction Overhead of Type-Classes}
\label{sct:type-classes-removal}

Type-classes are omnipresent in everyday programming as they allow abstraction over
 generic parameters (\eg, \code{Numeric} abstracts over numeric values). Unfortunately,
 type-classes introduce \emph{dynamic dispatch} on every call~\cite{rompf_optimizing_2013} and,
 thus, impose a performance penalty. Type-classes are in most of the cases statically known. Here
 we show how with \tool we can remove all abstraction overheads of type classes.

In Scala, type classes are implemented with objects and implicit parameters~\cite{oliveira_type_2010}.
In \figref{fig:numeric}, we define a \code{trait Numeric} serves as an interface for
all numeric types. Then we define a concrete implementation of \code{Numeric} for
type \code{Double} (\code{DoubleNumeric}). The \code{DoubleNumeric} is than passed
as an implicit argument \code{dnum} to all methods that use it (\eg, \code{zero}).

\begin{figure}
\begin{listingtiny}
object Numeric {
  implicit def dnum: Numeric[Double]@ct =
    ct(DoubleNumeric)
  def zero[T](implicit num: Numeric[T]@ct): T =
    num.zero
}

trait Numeric[T] {
  def plus(x: T, y: T): T
  def times(x: T, y: T): T
  def zero: T
}

object DoubleNumeric extends Numeric[Double] {
  def plus(x: Double, y: Double): Double = x + y
  def times(x: Double, y: Double): Double = x * y
  def zero: Double = 0.0
}
\end{listingtiny}
\caption{\label{fig:numeric} Removing abstraction overheads of type classes.}
\end{figure}

When \code{zero} is applied first the implicit argument (\code{dnum}) gets
inlined due to the \code{ct} annotation of the return type, then the function \code{zero} gets
inlined. Since \code{dnum} returns a compile-time view of \code{DoubleNumerc}
the method \code{zero} on \code{dnum} is evaluated at compile time. The constant \code{0.0} is
promoted to \code{ct} since \code{DoubleNumeric} is a compile time view. Finally the \code{ct(0.0)} result
is coerced to a dynamic value by the signature of \code{Numeric.zero}. The
compile-time execution is shown in the following snippet

\begin{lstparagraph}
Numeric.zero[Double]
  $\hookrightarrow$ Numeric.zero[Double](DoubleNumeric)
  $\hookrightarrow$ ct(DoubleNumeric).zero
  $\hookrightarrow$ (ct(0.0): Double)
  $\hookrightarrow$ 0.0
\end{lstparagraph}

\subsection{Inner Product of Vectors}
\label{sct:dot-product}

Here we demonstrate how the introductory example (\sct{sct:introduction}) is
partially evaluated through staging. We start with the desugared \code{dot}
function~(\ie, all implicit operations are shown):

\begin{lstparagraph}
 def dot[V](v1: Vector[V]@ct, v2: Vector[V]@ct)
  (implicit num: Numeric[V]@ct): V =
  (v1 zip v2).foldLeft(zero[V](num)) {
    case (prod, (cl, cr)) => prod + cl * cr
  }
\end{lstparagraph}

Function \code{dot} is generic in the type of vector elements. This will reflect
upon the staging annotations as well (\code{ct} and \code{static}). When we apply the
\code{dot} function with static arguments we will get the vector with static elements back:

\begin{lstparagraph}
dot[Double@static](
  ct(Vector)(2.0, 4.0), ct(Vector)(1.0, 10.0))(
  Numeric.dnum)
$\hookrightarrow$
  (ct(Vector)(2.0, 4.0) zip ct(Vector)(1.0, 10.0))
    .foldLeft(ct(0.0)) {
      case (prod, (cl, cr)) => prod + cl * cr
    }
$\hookrightarrow$ (2.0 * 1.0 + 4.0 * 10.0): Double@static
\end{lstparagraph}

When \code{dot} is evaluated with the \code{ct} elements the last step will further
execute to a single compile-time value that can further be used in compile-time computations:
\begin{lstparagraph}
dot[Double@ct](
  ct(Vector)(ct(2.0), ct(4.0)),
  ct(Vector)(ct(1.0), ct(10.0)))(Numeric.dnum)
$\hookrightarrow$ ct(2.0) * ct(1.0) + ct(4.0) * ct(10.0)
$\hookrightarrow$ 42.0: Double@ct
\end{lstparagraph}



\section{Discussion}
\label{sct:discussion}

% State the three possibilities.
To distinguish terms executed at compile-time from terms executed at runtime with type annotations we have the following possibilities:
\begin{enumerate}
\item Annotate types of all terms that should be executed at runtime. Here all types analyzed LMS and realized that this is not an option.


\item Annotate types of terms that should be executed at runtime but introduce scopes (e.g., method bodies) for which this rule applies.
In this way we would avoid annotating types of all runtime terms. This approach is taken by MacroML where
macro functions are executed at compile time and quoted terms are executed at runtime. First approach is, also,
a special case of this approach where there is a single scope for the whole language.

\item Annotate types of terms that are executed at compile time. This approach is used with \tool and annotated types are
called compile-time views.
\end{enumerate}

In \tool we decided to annotate types of terms that are executed at compile time. Compared to
the first solution our approach takes requires less annotations. We analyzed
% TODO Count annotations in OptiML

Compared to the second approach our solution is simpler to comprehend and communicate. In the second approach there are
two things that users need to understand when reasoning about staged programs: \emph{i)} where does
the compile time scope start, and \emph{ii)} which terms are annotated. With \tool the comprehension
is simple: terms whose types are annotated with \code{ct} are executed at compile time.

% The number of annotations if it is a mostly manipulate code snippets.
It is also interesting to the second and third approaches. Here the number of annotations
depends on the program. If the programs are mostly partially evaluated the second approach
is better. These category of programs could also be regarded as code generators as most of the code
is executed at compile time and produces large outputs. When programs are comprised of mostly
runtime values the approach of \tool requires less annotations.

\section{Limitations}
\label{sct:limitations}

\begin{itemize}
\item Interaction with type variables.
\item Type variables.
\item Type annotations and overloading and implicit search.
\item Can not inherit from a compile time view.
\end{itemize}


\section{Related Work}
\label{sct:related-work}

% Staging LMS Type Directed partial evaluation
MetaOCaml~\cite{taha_multi-stage_1997,calcagno2003implementing} is a staging extension
 for OCaml. It uses quotation to determine the stage in which the term is executed. Types of quoted terms are annotated
 to assure cross-stage persistence. Staging in MetaOCaml starts at host language runtime and
 can not express compile-time computations. Further, operations on annotated types
 do not get automatically promoted to the adequate stage of computation as with compile-time views.
 Finally, there are no implicit conversions so all stage promotions of terms must be explicit.

% MacroML
MacroML~\cite{ganz2001macros} is a language that translates macros into MetaML staging executed
 at compile time to provide a ``clean'' solution for macros. In MacroML, within
 the \code{let mac} construct function parameters can be annotated as an early stage computation. These parameters
 can then be used in escaped terms, \ie, terms scheduled to execute at compile time. Unlike \tool, MacroML
 uses escapes and early parameters to mark terms scheduled for to execute at compile time. Within
 escapes terms scheduled for runtime again need to be marked with brackets. This kind of dual annotations are
 not required as compile-time views are automatically promoted to runtime terms.

% LMS Type Directed partial evaluation
In LMS~\cite{rompf2012lightweight} terms that are annotated with \code{Rep} types will be executed at
 the stage after runtime compilation. Therefore, LMS can not directly be used for compile time computation. Furthermore,
 LMS requires additional reification logic and code generation for all \code{Rep} types.

 Programming language Idris~\cite{brady2010scrapping} introduces the \code{static} annotation
  on function parameters to achieve partial evaluation. Annotation \code{static} denotes
  that the term is statically known and that all operations on that term should
  be executed at compile-time. However, since \code{static} is placed on terms rather
  then types, it can mark only \emph{whole terms} as static. This restricts the number
  of programs that can be expressed, \eg, we could not express that vectors in the
  signature of \emph{dot} are static only in size. Finally, information about \code{static}
  terms can not be propagated through return types of functions so \code{static}
  in Idris is a partial evaluation construct, i.e., it hints that partial evaluation
  should be applied if function arguments are static.

% Specialization Scenarios
% TODO should we \cite{le2004specialization}

% Hybrid Partial Evaluation
Hybrid Partial Evaluation~(HPE)~\cite{shali2011Hybrid} is a technique for partial evaluation that
 does not perform binding time analysis (similarly to online partial evaluators) but relies on the user
 provided annotation \code{CT}\footnote{Name \code{ct} in \tool is inspired by hybrid partial evaluation.}.
 HPE implementations exist for both Java and Scala~\cite{sherwany2015refactoring}.
 Although, \code{CT} is used for partial evaluation, it does not affect typing of user programs. Furthermore,
 behavior of \code{CT} in context of generics is not described. \tool can be seen
 as statically typed version of hybrid partial evaluation with support for parametric polymorphism.
 Due to the support for parametric polymorphism \tool can express compile-time data structures with
 dynamic data.

% Forge
Forge~\cite{forge}, by Sujeeth et al., uses a DSL to declare a specification of the libraries.
 Forge then generates both unannotated and annotated code based on the specification.
 Their language also supports generating staged code (comprised of terms different from multiple stages).
 Forge specification and code generation supports only a subset of Scala guided towards the
 Delite~\cite{brown_heterogeneous_2011,composition-ecoop2013} framework.

% Yin-Yang
The Yin-Yang framework, by Jovanovic et al.~\cite{yin-yang}, solves the problem
 of code duplication by generating reification and code generation logic based on Scala code of existing types.
 With their approach there is no code duplication for the supported language features. However, not all of the
 Scala language is supported and all generated terms are generated for the next stage, thus,
 making a stage distinction is impossible.
