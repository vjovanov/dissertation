%!TEX root = ../phd.tex
\part{Removing Abstraction Overheads in Shallow Domain-Specific Languages}

\chapter{Introduction: Different Approaches to Removing Abstraction Overheads in Direct DSLs}
\label{sec:introdution-partial-evaluation}

\chapter{Polyvariant Staging}
\label{sec:polyvariant-staging}
 \section{Introduction}

 \emph{Multi-stage programming} (or \emph{staging}) is a meta-programming technique
  where compilation is separated in multiple \emph{stages}. Execution of each
  stage outputs code that is executed in the \emph{next stage} of compilation. The first
  stage of compilation happens at the \emph{host language} compile time, the second
  stage happens at the host language runtime, the third stage happens at runtime of
  runtime generated code, etc. Different stages of compilation can be executed in the same
  language~\cite{taha_multi-stage_1997,nielson2005two} or in different languages~\cite{brown_heterogeneous_2011,devito2013terra}.
  In this work we will focus on staging where all stages are in the same language and that, through static typing, assures that terms in the next stage are well typed.

  Notable staging systems in statically typed languages are
  MetaOCaml~\cite{taha_multi-stage_1997,calcagno2003implementing}
  and LMS~\cite{rompf2012lightweight}. These systems were successfully applied as a
  \emph{partial evaluatior}~\cite{jones1993partial}: for removing abstraction
  overheads in high-level programs~\cite{carette2005multi,rompf2012lightweight},
  for domain-specific languages~\cite{czarnecki_dsl_2004,jonnalagedda2014staged,taha2004gentle}, and for converting language
  interpreters into compilers~\cite{lancet,futamura1999partial}. Staging originates
  from research on two-level~\cite{nielson2005two,davies1996temporal} and multi-level~\cite{davies1996modal} calculi.

 We show an example of how staging is used for partial evaluation of a function
 for computing the inner product of two vectors\footnotemark[1]:\begin{lstparagraph}
def dot[T:Numeric](v1: Vector[T], v2: Vector[T]): T =
  (v1 zip v2).foldLeft(zero[T]) {
    case (prod, (cl, cr)) => prod + cl * cr
  }
 \end{lstparagraph}

In function \code{dot}, if vector sizes are constant, the inner product can
 be partially evaluated into a sum of products of vector components. To achieve partial evaluation,
 we must communicate to the staging system that operations on values of vector components
 should be executed in the next stage. The compilation stage
 in which a term is executed is determined by \emph{code quotation}~(in MetaOCaml)
 or by parametric types \code{Rep}~(in LMS). In LMS we denote that the vector size
 is statically known is achieved by annotating only vector elements with
 a \code{Rep} type\footnotemark[2]:\begin{lstparagraph}
def dot[T:Numeric]
  (v1: Vector[Rep[T]], v2: Vector[Rep[T]]): Rep[T]
 \end{lstparagraph}

Here the \code{Rep} annotations on \code{Rep[T]} denote that elements of vectors will be known only in the next stage (in LMS, this is a stage after run-time compilation). After run-time compilation \code{zip},
 \code{foldLeft}, and pattern matching inside the closure will not exist in the \emph{residual} program
 as they were evaluated in the previous stage of compilation (host language runtime). Note that in
 LMS unannotated code is always executed during host-language runtime
 and type-annotated code is executed after run-time compilation.

%  Monovariance
{\bf Stage monovariance.} The \code{dot} function in LMS is monovariant: it can only
 accept vectors that are statically known but their elements are dynamic. In order
 to support both versions the author of \code{dot} would have the body of
 \code{dot} with slightly different stage annotations: \begin{lstparagraph}
def dot[T:Numeric](v1: Vector[T], v2: Vector[T]): T
 \end{lstparagraph}

Designers of \emph{binding-time analysis} for offline partial-evaluators had the
same difficulties with monovariance. First solutions duplicate
code~\cite{rytz1992polyvariant} to achieve polyvariance. Duplication is  much
less troublesome with partial evaluation as users do not write the duplicate
code.  Henglein and Mossin introduce polymorphic binding-time
analysis~\cite{henglein1994polymorphic} to avoid code duplication, which was further refined to parametric
polymorphism~\cite{heldal2001binding}, and recursion and subtyping~\cite{dussart1995polymorphic}. Although effective,
these approaches are not used for staging but for offline partial evaluation. Finally, Ofenbeck et
al.~\cite{ofenbeck2013spiral} propose a solution to this problem based on type
classes and higher-kinded types. Their solution requires additional annotations
and implicit parameters in the type signature of polyvariant methods.

% Type annotations, and thus all operations continue to work!
{\bf Code Duplication.} Staging systems based on type annotations (e.g., LMS and type-directed
partial evaluation~\cite{danvy1999type}) inherently require code duplication as,
a priory, no operations are defined on \code{Rep} annotated types. For example,
in the LMS version of the \code{dot} function, all numeric types (\ie, \code{Rep[Int]}, \code{Rep[Double]}, etc.)
must be re-implemented in order to typecheck the programs and achieve code generation
for the next stage.

Sujeeth et al.~\cite{forge} and Jovanovic et al.~\cite{yin-yang}
 propose generating code for the next stage computations based on
 a language specification. These approaches solve the problem,
 but they require writing additional specification for the libraries,
 require a large machinery for code generation,
 and support only restricted parts of Scala.

{\bf Staging at host language compile time.} How can we use type based staging for programs whose values are statically
 known at the host language compile-time (the first stage)? Existing staging frameworks
 treat unannotated terms as runtime values of the host language and annotated terms as
 values in later stages of compilation. Even if we would take that the first stage is executed
 at the host language compile time, we would have to annotate all run-time values.
 Annotating all values is cumbersome since host language run-time values comprise
 the majority of user programs~(\sct{sct:discussion}).

\footnotetext[1]{All code examples are written in \emph{Scala}. It is necessary to
know the basics of Scala to comprehend this paper.}
\footnotetext[2]{In this work we use LMS as a representative of type-based staging systems.}

MacroML~\cite{ganz2001macros} expresses macros as two-stage computations that start executing from host language compile time.
 In MacroML, parameters of macros can be annotated as an early stage computation. These parameters
 can then used in escaped terms for compile-time computation. Terms scheduled for runtime execution,
 withing the escaped terms, again need to be quoted with brackets. This,
 in effect, imposes quotation for both escaping and brackets which requiring additional effort.


{\bf Polymorphic binding-time information.} The main idea of this paper is to use polymorphic binding-time analysis:
 \emph{i)} to enable stage polivariant code through bounded parametric polymorphism, and \emph{ii)} to allow reusing existing
 data types in different stages of computation without code duplication.

With bounded parametric polymorphism we can make functions stage polyvariant in their arguments by replacing them with
type parameters upperbounded by their original type. For example, polymorphic power function is defined as:\begin{lstparagraph}
@ct def pow[V <: Long](base: Long, exp: V): Long
\end{lstparagraph}

\emph{Annotated types} denote terms whose instances and non-polymorphic fields are known, and
 whose methods can be executed at in the \emph{previous stage} (i.e., compile time). We call annotated types \emph{compile-time views}
 of existing data types. Types are promoted to their compile-time views with type qualifiers~\cite{foster1999theory} expressed with
 the \code{@ct} annotation (\eg, \code{Int@ct}). In terms, statically known terms can be promoted
 their compile time duals with the function \code{ct}. By having two views of the same type
 we obviate the need for introducing reification and code generation logic for existing types.

With compile-time views, to require that vectors \code{v1} and \code{v2} are
 static and to partially evaluate the function, a programmer needs to make
 a simple modification of the \code{dot} signature:\begin{lstparagraph}
def dot[V: Numeric@ct]
  (v1: Vector[V]@ct, v2: Vector[V]@ct): V
\end{lstparagraph}

Since, vector elements are stage polymorphic the result
 of the function can be a dynamic value, or a compile-time view
 that can be further used for compile-time computations. The binding time of
 the return type of \code{dot} will match the binding time of vector elements:\begin{lstparagraph}
  // [el1, el2, el3, el4] are dynamic decimals
  dot(Vector(el1, el2), Vector(el3, el4))
    $\hookrightarrow$ (el1 * el3 + el2 * el4): Double

  // ct promotes static terms to compile-time views
  dot(Vector(ct(2), ct(4)), Vector(ct(1), ct(10)))
    $\hookrightarrow$ 42: Double@ct
\end{lstparagraph}


In this paper we contribute to the state of the art:
\begin{itemize}
% TODO links
 \item By using bounded parametric polymorphism as a vehicle to succinctly
   achieve polyvariant staging. This allows writing polyvariant stage programs
   without code duplication or additional language features. Stage annotations
   passed through type parameters are used by underlying polymorphic binding-time
   analysis to determine the correct binding-times.

 \item By obviating the need for reification and code generation logic in type based staging systems. The same binding-time
   analysis is to allow types defined in monovariant way to be used in multiple stages.

 \item By introducing compile-time views~(\sct{sct:interface}) as means to achieve
  type safe type based two-stage programming starting from host language compile time.

 \item By demonstrating the usefulness of compile-time views in four case
  studies (\sct{sct:case-studies}): inlining, partially evaluating recursion,
  removing overheads of variable argument functions, and removing overheads of
  type-classes~\cite{wadler1989make,hall_type_1996,oliveira_type_2010}.

\end{itemize}

We have implemented a staging extension for Scala \ct.
 \ct has a minimal interface (\sct{sct:interface}) based on type annotations.
 We have evaluated performance gains and the validity of \ct on all case
 studies~(\sct{sct:case-studies}) and compared them to LMS. In all benchmarks (\sct{sct:ct-evaluation})
 our evaluator performs the same as LMS and gives significant performance gains compared to original programs.


\section{\ct Interface}
\label{sct:interface}

In this section we present \ct, a staging extension for Scala based polymorphic binding-time analysis.
 \ct is a compiler plugin that executes in a phase after the
 Scala type checker. The plugin takes as input typechecked Scala programs and uses
 type annotations~\cite{odersky_1996_putting} to track and verify information about the biding-time
 of terms. It supports only two stages of compilation: host language compile-time
 (types annotated with \code{@ct}) and host language run-time (unannotated code).

To the user, \ct exposes a minimal interface (\figref{fig:interface}) with
a single annotation \code{ct} and a single function \code{ct}.

\begin{figure}
\begin{listingtiny}
package object scalact {
  final class ct extends StaticAnnotation

  def ct[T](body: => T): T = ???
}
\end{listingtiny}
\caption{Interface of \ct.}
\label{fig:interface}
\end{figure}

\smartparagraph{Annotation \code{ct}} is used on types~(\eg,
\code{Int@ct}) to promote them to their compile-time views. The
annotation is integrated in the Scala's type system and, therefore, can be
arbitrarily nested in different variants of types.

Since operations on compile-time views should be executed at compile time, the \code{ct} type
can be viewed as the original type whose instance is known at compile time with additional
 methods whose non-generic method parameters and result types also become compile-time views (if possible).
 Generic parameters remain unchanged as their binding time is defined during corresponding
 type application. Table \ref{tbl:ct-type} shows how the \code{@ct} annotation can be
 placed on types and how it affects method signatures of additional methods. Note that methods are not
 in fact added to the type but the binding-time polymorphic behavior of original methods can be observed
 as additional methods.

\begin{table*}[t]
\caption{Compile-time views of types and additional methods that will be available to the user.}
\label{tbl:ct-type}
\centering
\begin{tabularx}{\linewidth}{ X X X X }
\toprule

  Annotated Type              & \ &  Type's Method Signatures                          &  \\
  \code{Int@ct}               & \ &  \code{+(rhs: Int@ct): Int@ct}                     &  \\
  \code{Vector[Int]@ct}       & \ &  \code{map[U](f: (Int => U)@ct): Vector[U]@ct}     &  \\
                              & \ &  \code{length: Int@ct}                             &  \\
                              & \ &  \code{hashCode: Int}                              &  \\
  \code{Vector[Int@ct]@ct}    & \ &  \code{map[U](f: (Int@ct => U)@ct): Vector[U]@ct}  &  \\
                              & \ &  \code{length: Int@ct}                             &  \\
                              & \ &  \code{hashCode: Int@ct}                           &  \\
  \code{Map[Int@ct, Int]@ct}  & \ &  \code{get(key: Int@ct): Option[Int]@ct}           &  \\

\bottomrule
\end{tabularx}
\end{table*}

 In \tabref{tbl:ct-type}, on \code{Int@ct} both parameters and result types of all
 methods are also compile-time views. On the other hand, \code{Vector[Int]@ct} has parameters
 of all methods transformed except the generic ones. In effect, this, makes higher order combinators of \code{Vector}
 operate on dynamic values, thus, function \code{f} passed to \code{map} accepts
 the dynamic value as input. Note that \code{hashCode} of \code{Vector[Int]@ct} returns a dynamic value--this is
 due to its implementation that internally operates on generic values that are dynamic.
 Type \code{Vector[Int@ct]@ct} has all additional methods promoted to compile-time. The return type of the function \code{map} on \code{Vector[Int@ct]@ct}
 can still be either dynamic or a compile-time view due to the type parameter \code{U}.

Annotation \code{ct} can also be used to achieve inlining of statically known methods and functions.
 This is achieved by putting the annotation of the method/function\footnote{This is not the first time
that inlining is achieved through partial evaluation~\cite{monnier2003inlining}.}
 definition:\begin{lstparagraph}
 @ct def dot[T: Numeric]
  (v1: Vector[T], v2: Vector[T]): T
\end{lstparagraph}
Annotated methods will have an annotated method type\begin{lstparagraph}
((v1: Vector[T], v2: Vector[T]) => T)@ct
\end{lstparagraph} which can not be written by the users.

% Binding time is inferred for type parameters.
\smartparagraph{Function \code{ct}} is used at the term level
 for promoting literals, modules, and methods/functions into their compile-time views.
 \tabref{tbl:ct-term} shows how different types of terms are promoted to their
 compile-time views. An exception for promoting terms to compile-time views is the
 \code{new} construct. Here we use the type annotation on the constructed type.

\begin{table*}[t]
\caption{Promotion of terms to their compile-time views.}
\label{tbl:ct-term}
\centering
\begin{tabularx}{\linewidth}{ X X }
\toprule

  Promoted Term        \quad \quad \quad & Term's Type                      \\
  \code{ct(Vector)(1, 2, 3)            } & :\code{Vector[Int]@ct        }  \\
  \code{ct(Vector)(ct(1), ct(2), ct(3))} & :\code{Vector[Int@ct]@ct     }  \\
  \code{ct((x: Int@ct) => x)           } & :\code{(Int@ct => Int@ct)@ct }  \\
  \code{ct((x: Int) => x)              } & :\code{(Int => Int)@ct       }  \\
  \code{new (::@ct)(1, Nil)            } & :\code{(::[Int])@ct          }  \\
  \code{new (::@ct)(ct(1), ct(Nil))    } & :\code{(::[Int@ct])@ct       }  \\

\bottomrule
\end{tabularx}
\end{table*}

\subsection{Well-Formedness of Compile-Time Views}
\label{sct:wf-ctv}

% Nice description of csp and pointer to the right paper.
Earlier stages of computation can not depend on values from later stages. This property---defined as \emph{cross-stage persistence}~\cite{taha_multi-stage_1997,westbrook2010mint}---imposes that all operations on compile-time views must be known at compile time.

To satisfy cross-stage persistence \ct verifies that binding time of composite
 types~(\eg, polymorphic types, function types, record types, etc.) is always
 a subtype of the binding time of their components. In the following example,
 we show malformed types and examples of terms that are inconsistent:\begin{lstparagraph}
xs: List[Int@ct]     => ct(Predef).println(xs.head)
fn: (Int@ct=>Int@ct) => ct(Predef).println(fn(ct(1)))
\end{lstparagraph}

In the first example the program would, according to the semantics of \code{@ct}, print a head of the list at compile time.
 However, the head of the list is known only in the runtime stage. In the second example the program should
 print the result of \code{fn} at compile time but the body of the function will
 be known only at runtime. By causality such examples are not possible.

% Examples on classes
On functions/methods the \code{ct} annotation requires that function/method bodies are known at compile-time.
 Otherwise, inlining of such functions/methods would not be possible at compile-time. In Scala,
 method bodies are statically known in objects and classes with final methods, thus, the \code{ct}
 annotation is only applicable on such methods.

\subsection{Minimizing the Number of Annotations}
\label{sct:implicits}

One of the design goals of \ct is to minimize the number of superfluous staging annotations. We achieve
 this by implicitly adding annotations and term promotions that would with high probability be added by the user.

{\bf Adding \code{ct} to Functions.} Due to cross-stage persistence if a single parameter of
a function is annotated with \code{ct} the whole function must also be \code{ct}. Since forgetting
this annotation would only result in an error we implicitly add
the \code{ct} annotation when at least one parameter type or the result type is marked as \code{ct}.

{\bf Implicit Conversions.} If method parameters require the compile-time view of a type the corresponding arguments
 in method application would always have to be promoted to \code{ct}.
 In some libraries this could require an inconveniently large number of annotations.

To minimize the number of required annotations we introduce \emph{implicit conversions}
 from certain \code{static} terms to \code{ct} terms. Note that we use a custom mechanism for
 implicit conversions based on type annotations. This is necessary as Scala implicit conversions
 are oblivious to type annotations (\sct{sct:limitations}).

The conversions support translation of language literals, direct class constructor calls with static arguments, and static method
 calls with static arguments into their compile-time views. Since our compile-time evaluator does
 not use Asai's~\cite{asai2002binding,sumii2001hybrid} method to keep track of
 the value of each static term, we disallow implicit conversions of terms with static variables.

For example, for a factorial function \begin{lstparagraph}
def fact(n: Int@ct): Int@ct =
  if (n == 0) 1 else fact(n - 1)
 \end{lstparagraph} we will not require promotions of literals \code{0}, and \code{1}. Furthermore,
 the function can be invoked without promoting the argument into it's compile-time view:\begin{lstparagraph}
fact(5)
  $\hookrightarrow$ 120
 \end{lstparagraph}

Without implicit conversions the factorial functions would be more verbose\begin{lstparagraph}
@ct def fact(n: Int@ct): Int@ct =   if (n == ct(0)) ct(1)
else fact(n - ct(1))
\end{lstparagraph}

as well as each function application (\code{fact(ct(5))}).

Implicit conversions are safe in all cases except when the user should be warned about potential
code explosion. For example, a user could accidentally call \code{fact} with a very large number and
cause code explosion without even knowing that staging is happening. If \code{ct} was required
it would remind the users about the potential problems. In the design of \ct we decided to
prefer less annotations over code-explosion prevention.


\section{Polymorphic Binding-Time Analysis} \label{sct:bta}


% Define binding time abstractions

% Define polymorphic binding times

% Define the translation

\subsection{Nominal Types and Subtyping}
\label{sct:nominal-types}

\subsection{Compile-Time Evaluation}
\label{sct:evaluation}

\subsection{Mutable State}
\label{sct:mutable-state}
% TODO arrays and variables
% TODO attachments
% TODO mutable state not for now
% TODO emulate arrays as mutable lists


\section{Case Studies}
\label{sct:case-studies}

In this section we present selected use-cases for compile-time views that, at the
same time, demonstrate step-by-step the mechanics behind \ct. We start by inlining a simple function with staging
(\sct{sct:inlining}), then do the canonical staging  example of the integer power function
(\sct{sct:recursion}), then we demonstrate how variable argument functions can
be  desugared into the core functionality (\sct{sct:varargs}). Finally, we
demonstrate how the abstraction overhead of the \code{dot} function and all
associated type-class related abstraction an be removed (\sct{sct:dot-product}).

\subsection{Inlining Expressed Through Staging}
\label{sct:inlining}

Function inlining can be expressed as staged computation~\cite{monnier2003inlining}.
 Inlining is achieved when a statically known function body is applied with symbolic
 arguments. In \ct we use the \code{ct} annotation on functions and methods to achieve inlining:\begin{lstparagraph}
@ct def zero[T](implicit num: Numeric[T]) = num.zero

zero[Double]
  $\hookrightarrow$ num.zero
\end{lstparagraph}


\subsection{Recursion}
\label{sct:recursion}

The canonical example in staging literature is partial evaluation of the power function
 where exponent is an integer:
\begin{lstparagraph}
def pow(base: Double, exp: Int): Double =
  if (exp == 0) 1 else base * pow(base, exp - 1)
\end{lstparagraph} When the exponent (\code{exp}) is statically known this function can be partially
evaluated into \code{exp} multiplications of the \code{base} argument, significantly
improving performance~\cite{calcagno2003implementing}.

With compile-time views making \code{pow} partially evaluated requires adding only one annotation:

\begin{lstparagraph}
def pow(base: Double, exp: Int@ct): Double =
  if (exp == 0) 1 else base * pow(base, exp - 1)
\end{lstparagraph}

% TODO cite infinite recursion
To satisfy cross-stage persistence (\sct{sct:wf-ctv}) the \code{pow} must be \code{@ct}.
This annotation is automatically added by \ct as described in \sct{sct:implicits}. In the example,
 the \code{ct} annotation on \code{exp} requires that the function must be called with
 a compile-time view of \code{Int}. \ct ensures that the definiton of the \code{pow} function
 does not cause infinite recursion at compile-time by invoking the power function
 only when the value of the \code{ct} arguments is known.

 The application of the function \code{pow} with a constant
 exponent produces:

\begin{lstparagraph}
pow(base, 4)
  $\hookrightarrow$ base * base * base * base * 1
\end{lstparagraph}

Constant 4 is promoted to \code{ct} by the implicit conversions (\sct{sct:implicits}).

\subsection{Variable Argument Functions}
\label{sct:varargs}

% Variable argument functions
Variable argument functions appear in widely used languages like Java, C\#, and Scala.
 Such arguments are typically passed in the function body inside of the data structure
 (\eg \code{Seq[T]} in Scala). When applied with variable arguments the size of the
 data-structure is statically known and all operations on them can be partially
 evaluated. However, sometimes, the function is called with arguments of dynamic size.
 For example, function \code{min} that accepts multiple integers\begin{lstparagraph}
def min(vs: Int*): Int = vs.tail.foldLeft(vs.head) {
  (min, el) => if (el < min) el else min
}
\end{lstparagraph}can be called either with statically known arguments
 (\eg, \code{min(1,2)}) or with dynamic arguments:\begin{lstparagraph}
val values: Seq[Int] = ... // dynamic value
min(values: _*)
\end{lstparagraph}

Ideally, we would be able to achieve partial evaluation if the arguments are of statically
known size and avoid partial evaluation in case of dynamic arguments. To this end we translate
the method \code{min} into a partially evaluated version and a dynamic version. The call to these
methods is dispatched, at compile-time, by the \code{min} method which checks if
arguments are statically known. Desugaring of \code{min} is shown in \figref{fig:min}.

\begin{figure}
\begin{listingtiny}
def min(vs: Int*): Int = macro
  if (isVarargs(vs)) q"min_CT(vs)"
  else q"min_D(vs)"

def min_CT(vs: Seq[Int]@ct): Int =
  vs.tail.foldLeft(vs.head) { (min, el) =>
    if (el < min) el else min
  }

def min_D(vs: Seq[Int]): Int =
  vs.tail.foldLeft(vs.head) {
    (min, el) => if (el < min) el else min
  }
\end{listingtiny}
\caption{Function \code{min} is desugared into a \code{min} macro that based on the
binding time of the arguments dispatches to the partially evaluated version (\code{min_CT})
for statically known varargs or to the original min function for dynamic arguments \code{min_D}.}
\label{fig:min}
\end{figure}

\subsection{Removing Abstraction Overhead of Type-Classes}
\label{sct:type-classes-removal}

Type-classes are omnipresent in everyday programming as they allow abstraction over
 generic parameters (\eg, \code{Numeric} abstracts over numeric values). Unfortunately,
 type-classes introduce \emph{dynamic dispatch} on every call~\cite{rompf_optimizing_2013} and,
 thus, impose a performance penalty. Type-classes are in most of the cases statically known. Here
 we show how with \ct we can remove all abstraction overheads of type classes.

In Scala, type classes are implemented with objects and implicit parameters~\cite{oliveira_type_2010}.
In \figref{fig:numeric}, we define a \code{trait Numeric} serves as an interface for
all numeric types. Then we define a concrete implementation of \code{Numeric} for
type \code{Double} (\code{DoubleNumeric}). The \code{DoubleNumeric} is than passed
as an implicit argument \code{dnum} to all methods that use it (\eg, \code{zero}).

\begin{figure}
\begin{listingtiny}
object Numeric {
  implicit def dnum: Numeric[Double]@ct =
    ct(DoubleNumeric)
  def zero[T](implicit num: Numeric[T]@ct): T =
    num.zero
}

trait Numeric[T] {
  def plus(x: T, y: T): T
  def times(x: T, y: T): T
  def zero: T
}

object DoubleNumeric extends Numeric[Double] {
  def plus(x: Double, y: Double): Double = x + y
  def times(x: Double, y: Double): Double = x * y
  def zero: Double = 0.0
}
\end{listingtiny}
\caption{\label{fig:numeric} Removing abstraction overheads of type classes.}
\end{figure}

When \code{zero} is applied first the implicit argument (\code{dnum}) gets
inlined due to the \code{ct} annotation of the return type, then the function \code{zero} gets
inlined. Since \code{dnum} returns a compile-time view of \code{DoubleNumerc}
the method \code{zero} on \code{dnum} is evaluated at compile time. The constant \code{0.0} is
promoted to \code{ct} since \code{DoubleNumeric} is a compile time view. Finally the \code{ct(0.0)} result
is coerced to a dynamic value by the signature of \code{Numeric.zero}. The
compile-time execution is shown in the following snippet

\begin{lstparagraph}
Numeric.zero[Double]
  $\hookrightarrow$ Numeric.zero[Double](DoubleNumeric)
  $\hookrightarrow$ ct(DoubleNumeric).zero
  $\hookrightarrow$ (ct(0.0): Double)
  $\hookrightarrow$ 0.0
\end{lstparagraph}

\subsection{Inner Product of Vectors}
\label{sct:dot-product}

Here we demonstrate how the introductory example is partially evaluated through staging.
 We start with the desugared \code{dot} function~(\ie, all implicit operations are shown):

\begin{lstparagraph}
 def dot[V](v1: Vector[V]@ct, v2: Vector[V]@ct)
  (implicit num: Numeric[V]@ct): V =
  (v1 zip v2).foldLeft(zero[V](num)) {
    case (prod, (cl, cr)) => prod + cl * cr
  }
\end{lstparagraph}

Function \code{dot} is generic in the type of vector elements. This will reflect
upon the staging annotations as well (\code{ct} and \code{static}). When we apply the
\code{dot} function with static arguments we will get the vector with static elements back:

\begin{lstparagraph}
dot[Double@static](
  ct(Vector)(2.0, 4.0), ct(Vector)(1.0, 10.0))(
  Numeric.dnum)
$\hookrightarrow$
  (ct(Vector)(2.0, 4.0) zip ct(Vector)(1.0, 10.0))
    .foldLeft(ct(0.0)) {
      case (prod, (cl, cr)) => prod + cl * cr
    }
$\hookrightarrow$ (2.0 * 1.0 + 4.0 * 10.0): Double@static
\end{lstparagraph}

When \code{dot} is evaluated with the \code{ct} elements the last step will further
execute to a single compile-time value that can further be used in compile-time computations:
\begin{lstparagraph}
dot[Double@ct](
  ct(Vector)(ct(2.0), ct(4.0)),
  ct(Vector)(ct(1.0), ct(10.0)))(Numeric.dnum)
$\hookrightarrow$ ct(2.0) * ct(1.0) + ct(4.0) * ct(10.0)
$\hookrightarrow$ 42.0: Double@ct
\end{lstparagraph}




\section{Discussion}
\label{sct:discussion}

% State the three possibilities.
To distinguish terms executed at compile-time from terms executed at runtime with type annotations we have the following possibilities:
\begin{enumerate}
\item Annotate types of all terms that should be executed at runtime. Here all types analyzed LMS and realized that this is not an option.

\item Annotate types of terms that should be executed at runtime but introduce scopes (e.g., method bodies) for which this rule applies.
In this way we would avoid annotating types of all runtime terms. This approach is taken by MacroML where
macro functions are executed at compile time and quoted terms are executed at runtime. First approach is, also,
a special case of this approach where there is a single scope for the whole language.

\item Annotate types of terms that are executed at compile time. This approach is used with \ct and annotated types are
called compile-time views.
\end{enumerate}

To compare approach of \ct with the first approach we analyzed 817 functions
from the OptiML~\cite{sujeeth_optiml:_2011} DSL based on LMS. With the \ct scheme
OptiML would require more than 2x less annotations to implement.

Compared to the second approach our solution is simpler to comprehend and communicate. In the second approach there are
two things that users need to understand when reasoning about staged programs: \emph{i)} where does
the compile time scope start, and \emph{ii)} which terms are annotated. With \ct the comprehension
is simple: terms whose types are annotated with \code{ct} are executed at compile time.

% The number of annotations if it is a mostly manipulate code snippets.
It is also interesting to the second and third approaches. Here the number of annotations
depends on the program. If the programs are mostly partially evaluated the second approach
is better. These category of programs could also be regarded as code generators as most of the code
is executed at compile time and produces large outputs. When programs are comprised of mostly
runtime values the approach of \ct requires less annotations.


\section{Evaluation}
\label{sct:ct-evaluation}

In this section we evaluate the amount of code that is obviated with \ct compared to existing
type directed staging systems (\sct{sct:duplication}). Then we evaluate performance of
\ct compared to LMS and hand optimized code (\sct{sct:performance})

\subsection{Reduction in Code Duplication}
\label{sct:duplication}

Evaluating reduction of duplicated code (for reification and code generation) in type based
staging systems is difficult as the factor varies from program to program. To avoid benchmark dependent
results we instead calculate the lower bound on the duplication factor.

Given that we have a method on a type \code{T} whose body contains \code{n} lines of code (without
the method definition). To introduce the same method on an annotated type \code{Rep[T]} we need another
 method for reification which has at least 1 line of code. Then we need code generation
 logic, which, if we use the same language should not have less lines than the original method
 plus at least one line for matching the reified method. For method of $n$ lines
 we get a lower bound on the code duplication factor of:$$
 2n+3/n+1
$$
For single line methods ($n=0$) the factor is 3 and for large methods ($n\rightarrow\infty$) it converges to 2.

\subsection{Performance of Generated Code}

In this section we compare performance of \ct with LMS and original code. All benchmarks
are executed on an Intel Core i7 processor (4960HQ) working frequency of 2.6 GHZ with 16GB
of DDR3 with a working frequency of 1600 MHz. For all benchmarks we use Scala
2.11.5 and the HotSpot(TM) 64-Bit Server (24.51-b03) virtual machine. In all benchmarks
the virtual machine is warmed up, no garbage collection happens, and all reported numbers are
a mean of 5 measurements.

In \tabref{tbl:numbers} we show execution time normalized to original code for:
 \emph{i)} \code{pow(42.0, 10)},
 \emph{ii)} \code{min(a, b, c, d, e)},
 \emph{iii)} inner product of two statically known vectors of size 50,
 and \emph{iv)} the butterfly network of size 4 for fast Fourier transform \code{fft} (equivalent to code presented by Rompf and Odersky~\cite{rompf2012lightweight}). For all benchmarks the performance results are equivalent to LMS.

\label{sct:performance}
\begin{table}[ht]
\caption{Speedup of LMS and \ct compared to the naive implementation of the algorithms.}
\label{tbl:numbers}
\centering
\begin{tabularx}{\linewidth}{ X X X }
\toprule

  Benchmark                   &  LMS      &  \ct                             \\
  \code{pow}                  &    221.75 & 221.70                             \\
  \code{min}                  &    1.82   & 1.79                               \\
  \code{dot}                  &  246.08   & 246.08                             \\
  \code{fft}                  &  12.14    & 12.88                              \\

\bottomrule
\end{tabularx}
\end{table}


\section{Limitations}
\label{sct:limitations}

{\bf Type Annotations.} Using type annotations for annotating the compilation stage is not ideal. Type annotations are not fully
 integrated into the Scala language. Major drawbacks are that overloading resolution and implicit search are oblivious about annotations. For example, if two methods have the same signatures but different staging annotations the compiler will report an error. Implicit search will fail in two ways: \emph{i)} if two implicit functions with the same type are in scope but annotations differ the compiler will report an error about ambiguous implicits and \emph{ii)} if a method requires an implicit parameter with the \code{ct} annotation the compiler might provide an implicit argument without the annotation. These are not fundamental
 limitations and, in practice, stage polyvariance can be used to avoid difficulties with annotations.

{\bf Type Annotation Position.} Annotations in Scala can be used in many different positions and \ct supports only some of them. Annotation \code{ct} can not be used in following positions: \emph{i)} on classes, traits, and modules, \emph{ii) in the list of inherited classes and traits}, \emph{iii)} on the right hand side of the type variable definitions, and \emph{iv)} on all terms outside the method definitions (constructors, constructor arguments, etc.).

{\bf Access Modifiers.} Scala supports access modifiers of members. If methods that use \ct internally access \code{private} members of the class \ct will fail as all staged methods are inlined at the call site. Similar limitations exist with inlining functions in Scala. This problem could be circumvented inside Scala, however, the JVM will not allow this in the bytecode verification phase.

{\bf Code Explosion.} Annotation \code{ct} inlines all functions that are annotated and unrolls all loops on compile-time executed data structures. This can, for a staged function, lead to unnaceptable code explosion for some inputs while the code can behave regularly for other inputs. For example, calling \code{pow(0.5, 10000000)} on the function from \sct{sct:case-studies} will make unacceptably large code while \code{pow(0.5, 10)} will work as expected.

\subsection{Nominal Typing, Lower Bounds, and Higher-Kinded Types}
\label{sct:nominal-typing}


\section{Related Work}
\label{sec:related-work-staging}

% Staging LMS Type Directed partial evaluation
MetaOCaml~\cite{taha_multi-stage_1997,calcagno2003implementing} is a staging extension
 for OCaml. It uses quotation to determine the stage in which the term is executed. Types of quoted terms are annotated
 to assure cross-stage persistence. Staging in MetaOCaml starts at host language runtime and
 can not express compile-time computations. Further, operations on annotated types
 do not get automatically promoted to the adequate stage of computation as with compile-time views.
 Finally, there are no implicit conversions so all stage promotions of terms must be explicit.

% MacroML
MacroML~\cite{ganz2001macros} is a language that translates macros into MetaML staging executed
 at compile time to provide a ``clean'' solution for macros. In MacroML, within
 the \code{let mac} construct function parameters can be annotated as an early stage computation. These parameters
 can then be used in escaped terms, \ie, terms scheduled to execute at compile time. Unlike \ct, MacroML
 uses escapes and early parameters to mark terms scheduled for to execute at compile time. Within
 escapes terms scheduled for runtime again need to be marked with brackets. This kind of dual annotations are
 not required as compile-time views are automatically promoted to runtime terms.

% LMS Type Directed partial evaluation
In LMS~\cite{rompf2012lightweight} terms that are annotated with \code{Rep} types will be executed at
 the stage after runtime compilation. Therefore, LMS can not directly be used for compile time computation. Furthermore,
 LMS requires additional reification logic and code generation for all \code{Rep} types.

 Programming language Idris~\cite{brady2010scrapping} introduces the \code{static} annotation
  on function parameters to achieve partial evaluation. Annotation \code{static} denotes
  that the term is statically known and that all operations on that term should
  be executed at compile-time. However, since \code{static} is placed on terms rather
  then types, it can mark only \emph{whole terms} as static. This restricts the number
  of programs that can be expressed, \eg, we could not express that vectors in the
  signature of \emph{dot} are static only in size. Finally, information about \code{static}
  terms can not be propagated through return types of functions so \code{static}
  in Idris is a partial evaluation construct, i.e., it hints that partial evaluation
  should be applied if function arguments are static.

% Specialization Scenarios
% TODO should we \cite{le2004specialization}

% Hybrid Partial Evaluation
Hybrid Partial Evaluation~(HPE)~\cite{shali2011Hybrid} is a technique for partial evaluation that
 does not perform binding time analysis (similarly to online partial evaluators) but relies on the user
 provided annotation \code{CT}\footnote{Name \code{ct} in \ct is inspired by hybrid partial evaluation.}.
 HPE implementations exist for both Java and Scala~\cite{sherwany2015refactoring}.
 Although, \code{CT} is used for partial evaluation, it does not affect typing of user programs. Furthermore,
 behavior of \code{CT} in context of generics is not described. \ct can be seen
 as statically typed version of hybrid partial evaluation with support for parametric polymorphism.
 Due to the support for parametric polymorphism \ct can express compile-time data structures with
 dynamic data.

% Forge
Forge~\cite{forge}, by Sujeeth et al., uses a DSL to declare a specification of the libraries.
 Forge then generates both unannotated and annotated code based on the specification.
 Their language also supports generating staged code (comprised of terms different from multiple stages).
 Forge specification and code generation supports only a subset of Scala guided towards the
 Delite~\cite{brown_heterogeneous_2011,composition-ecoop2013} framework.

% Yin-Yang
\todo{what to do with this}
The Yin-Yang framework, by Jovanovic et al.~\cite{yin-yang}, solves the problem
 of code duplication by generating reification and code generation logic based on Scala code of existing types.
 With their approach there is no code duplication for the supported language features. However, not all of the
 Scala language is supported and all generated terms are generated for the next stage, thus,
 making a stage distinction is impossible.
